This paper consider query-based wireless sensor networks (WSN). Nodes are requested for information on demand. A design of such a network is presented and a prototype implementation is deployed. We study energy efficiency in such systems. We found that even while having the same cumulative sleep duration, the duration of each sleep cycle have an effect on energy consumption. This is due to overhead caused by polling at the end of each sleep cycle. Thus, we tackle the problem of maximizing sleep periods while ensuring a QoS requirement, namely delay. We explore a model that will enable us to calculate suitable sleeping times. We perform a simulation study and an experimental validation of our findings.
