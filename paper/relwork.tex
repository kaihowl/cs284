\section{Related Work} \label{sec:relwork}

A transactional framework for WSNs was proposed in \cite{2, 24}. Our framework is based on them. Specifically we will use the notion of one-time and persistent queries. Queries specify rules that include among others: (1) the duration of needed observations. (2) frequency. (3) upper/lower thresholds. (4) desired locations/sensors. and (5) QoS requirements. In \cite{2}, a hybrid storage system to maintain historical data is proposed; historical data should be sent out by the node to a central site where it is stored for later retrieval. We pursue a hybrid approach where a user can query a node directly or through the front-end. If a central node or entity was interested in maintaining the history of a specific node, it can poll it on demand or issue a persistent query. A persistent query tells a node to send collected data for a specific duration and rate, e.g., send me the temperature every ten minutes for the next seven days.

Energy conservation gained a lot of attention in the WSNs community \cite{1,5,6,7}; proposals to design MAC and routing protocols, in addition to topology control that are energy aware are examples of such work. We, however, consider a more practical approach; we consider network topology is a given. This is important since we cannot assume that we have control on the topology or movement of nodes. Also, we consider off-the-shelf hardware. Thus, we model our solution around the IEEE 802.15.4 standard and AODV protocol \cite{aodv}. Some other work also tried to model IEEE 802.15.4 networks \cite{8,9,10,4.11,12,13}. Some work even model the network for energy efficiency \cite{10,12,13}. Our work focuses on on-demand query-based query-based WSNs. This type of WSNs has also been treated for energy efficiency consideration \cite{14,25}. In~\cite{25}, the authors try to minimize energy consumption by sleeping. They develop a probabilistic model that trade-off between energy and data quality. Our model extends the work done in the literature of modeling query-based WSNs by the following. We consider the trade-off between QoS and energy efficiency. Also, we employ an M/G/1 model with vacations to capture the system and derive appropriate sleeping times given QoS requirements. Similar employment of this mathematical treatment is done in \cite{15} for optical networks. We use their approach and modify it to model IEEE 802.15.4 sleeping times.

%/* Review the system (Towards a sensor database is a good source \& tinyDB arch.). With an example of a query and platform/application. */
Sensor networks are being widely deployed for measurement, detection and surveillance applications.  Related work to data processing in a wireless sensor network is TinyDB, a query processing system for extracting information from a sensor network. TinyDB provides a SQL like interface, for each input request it collects data from the nodes, filters it and sends the aggregated result to a PC via power efficient network processing algorithm. Queried data consists of sensor attributes (e.g., location) as well as sensor data. A typical monitoring scenario involves aggregate queries or correlation queries. In order to optimize the system's  performance, result of these queries are partially aggregated at each intermediate node. In our system we have implemented an interface to query on the properties such as the sensor temperature and further grouped the results on the basis of floor numbers.

