\section{Introduction}\label{sec:Intro}

Wireless Sensor Networks (WSNs) use sensors to measure a wide variety of live data and actuators to manipulate their environments. Data are communicated between nodes and sinks. WSNs are self-configuring, inherently small, low-cost, and low-power. This makes it a suitable technology for use in monitoring and tracking applications. Many of the envisioned applications of sensor networks require a large-scale deployment that comprises thousands of sensor nodes, e.g., the Internet of things \cite{22} and sensor database systems \cite{2}. With a large number of sensor nodes, it becomes challenging to maintain all collected data in a centralized unit (or multiple centralized units). Moreover, the demand for data might vary from node to node. Keeping the node awake at all times is a waste of communication resources and will ultimately drain the nodes power prematurely. A energy-aware demand-driven transactional protocol is needed to mitigate this waste of resources. Many papers propose a distributed approach to data-management in sensor networks \cite{2,4.11,23}. Instead of having a central database collecting sensor readings from individual nodes, the data is stored and distributed over the nodes themselves. Queries then extract the necessary data from the sensor nodes. 

In a transactional distributed scheme the node needs to be alerted for any data queries. Thus, making the node awake (i.e., consuming energy) even if it is not sending or collecting data. A desirable property of WSNs is power-efficiency. The main objective of this work is to save power and prolong the life of nodes by making them consume as less power as possible. Having low sleep times will make a node handle incoming requests faster, since they are only waiting for a short period. However, a longer sleeping period will minimize the number of switches between sleeping and active modes. These switches incur a communication overhead; a newly awaken node must poll its parent node for requests. Furthermore, the waking time (the time required for a node to switch from sleeping to active) ranges between 2 to 13.2 milliseconds for XBEE series 2 nodes \cite{19}. Thus, we aim to maximize sleeping time while we maintain QoS requirements.

A transactional framework for WSNs was proposed \cite{2, 24}. Our framework is based on them. Specifically we will use the notion of one-time and persistent queries. Queries specify rules that include among others: (1) the duration of needed observations. (2) frequency. (3) upper/lower thresholds. (4) desired locations/sensors. and (5) QoS requirements. In \cite{2}, a hybrid storage system to maintain historical data is proposed; historical data should be sent out by the node to a central site where it is stored for later retrieval. We pursue a more a radical approach and propose to employ a fully distributed scheme. If a central node or entity was interested in maintaining the history of a specific node, it can poll it on demand or issue a persistent query. A persistent query tells a node to send collected data for a specific duration and rate, e.g., send me the temperature every ten minutes for the next seven days.

Energy conservation gained a lot of attention in the WSNs community \cite{1,5,6,7}; proposals to design MAC and routing protocols, in addition to topology control that are energy aware are examples of such work. We, however, consider a more practical approach; we consider network topology is a given. This is important since we cannot assume that we have control on the topology or movement of nodes. Also, we consider off-the-shelf hardware. Thus, we model our solution around the IEEE 802.15.4 standard and AODV protocol \cite{aodv}. Some other work also tried to model IEEE 802.15.4 networks \cite{8,9,10,4.11,12,13}. Some work even model the network for energy efficiency \cite{10,12,13}. Our work focuses on on-demand query-based transactional WSNs. This type of WSNs has also been treated for energy efficiency consideration \cite{14,25}. In~\cite{25}, the authors try to minimize energy consumption by sleeping. They develop a probabilistic model that trade-off between energy and data quality. Our model extends the work done in the literature of modeling transactional WSNs by the following. We consider the trade-off between QoS and energy efficiency. Also, we employ an M/G/1 model with vacations to capture the system and derive appropriate sleeping times given QoS requirements. Similar employment of this mathematical treatment is done in \cite{15} for optical networks. We use their approach and modify it to model IEEE 802.15.4 sleeping times.

The paper is divided as the following. \ldots