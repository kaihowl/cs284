\section{Introduction}\label{sec:Intro}

Wireless Sensor Networks (WSNs) use sensors to measure a wide variety of live data and actuators to manipulate their environments. Data is communicated between nodes and sinks. WSNs are self-configuring, inherently small, low-cost, and low-power. This makes it a suitable technology for use in monitoring and tracking applications. Many of the envisioned applications of sensor networks require a large-scale deployment that comprises thousands of sensor nodes, e.g., the Internet of things \cite{22} and sensor database systems \cite{2}. With a large number of sensor nodes it becomes challenging to maintain all collected data in a centralized unit (or multiple centralized units). Moreover, the demand for data might vary from node to node. Keeping the node awake at all times is a waste of communication resources and will ultimately drain the nodes power prematurely. An energy-aware demand-driven query-based protocol is needed to mitigate this waste of resources. Many papers propose a distributed approach to data-management in sensor networks \cite{2,4.11,23}. Instead of having a central database collecting sensor readings from individual nodes, the data is stored and distributed over the nodes themselves. Queries then extract the necessary data from the sensor nodes. 

In a query-based distributed scheme the node needs to be alert for any data queries. Thus, making the node awake (i.e., consuming energy) even if it is not sending or collecting data. A desirable property of WSNs is power-efficiency. The main objective of this work is to save power and prolong the life of nodes by making them consume as less power as possible. Having low sleep cycle times will make a node handle incoming requests faster, since they are only waiting for a short period. However, a longer sleeping cycle will minimize the number of switches between sleeping and active modes. These switches incur a communication overhead and therefore an energy waste; a newly awaken node must poll its parent node for requests. Furthermore, the waking time (the time required for a node to switch from sleeping to active) ranges between 2 to 13.2 milliseconds for XBEE series 2 nodes \cite{19}. Thus, we aim to maximize sleeping time while we maintain QoS requirements to minimize the number of switches.

In this paper we use a queueing system model to capture delay characteristics. We study this model to explore its suitability in capturing our system. Specifically, since we are having a two-way communication dynamic with packets generations interdependent to each other, it is not intuitively clear whether we can find a suitable solution to model our system. Our contribution is threefold:
\begin{itemize}
\item{We demonstrate the effect of maximizing the sleep cycle duration on energy consumption. The paper shows that even with the same cumulative sleep duration, the number of switches from active to sleep state incur a significant effect on energy.}
\item{We explore a queueing system model's capability in capturing a query-based WSN system. In particular, we perform an experimental evaluation of MAC layer dynamics to aid in developing the model.}
\item{We perform a simulation and experimental study of the system. Also, we demonstrate the validity of our findings.}
\end{itemize}

The paper is divided as follows. Section 2 details the motivation of our research, Section 3 outlines related work, Section 4 summarizes the technical background of the prototype, Section 5 describes the system design of the prototype, Section 6 proposes a model for QoS in our system, Section 7 evaluates the model both with the prototype and with simulations, Section 8 concludes with a summary and directions for future work.
