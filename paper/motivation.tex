\section{Motivation} \label{sec:motivation}

/* Write about the battery drain experiment and how a longer battery life can be achieved by having a longer sleep resp. poll time.*/


In our project we tackle one of the main challenges in WSNs, namely energy efficiency. Sensors have a limited amount of power. In the absence of any tasks, a node wastes its energy waiting for transmissions. Making a node sleep (i.e., consume less power) in inactive periods is a well-studied way to leverage unwanted use of power \cite{1}. The main objective of our work is to maximize sleeping periods while maintaining QoS requirements; in our study we will focus on delay and memory. A transactional query-based framework is considered for our implementation \cite{2}. Furthermore, we will use ZigBee and IEEE 802.15.4 \cite{3} as our system's infrastructure. The objective of our work is three-fold: First, we will deploy a query-based WSN using Arduino microcontrollers \cite{17} and XBEE modules \cite{18}. Second, we will implement the query-based scheme over QualNet simulator~\cite{16}. Using the deployment and simulation framework we can study the system for the effects of sleeping on QoS and energy consumption. Third, a mathematical model will be constructed to capture QoS and energy characteristics. From the model, a formula for determining maximum sleeping times is derived. The resulted formula will be tested in the deployment (as a proof-of-concept prototype) and in simulation (to investigate scalability).

The main contribution of our work is to produce a mathematical model that can be used to capture QoS and energy characteristics of WSNs. This model can then be used to derive suitable sleeping time when given certain QoS characteristics and network load. A queueing system \cite{21} can be used to model this problem. An M/G/1 queueing model with vacations \cite{20} is, we claim, suitable for such a problem.

Another challenge is to deploy and simulate a query-based WSN. A full deployment of transaction handling distributively in WSNs would include transaction processing and optimization, routing, aggregation, etc. \cite{2}. We will start with a simple query-based scheme in our deployments. However, we will design it so it would be possible to incrementally develop it to include more transactional features. 
