\section{System design}\label{sec:design}

\subsection{Front-end design}
/* This subsection on the front-end (server) that will take queries and process them and transform them to light-weight queries sent to the sensors */

\subsubsection{Query processing}
/* The process of breaking the query to simpler light-weight queries. How data is assembled from nodes etc */

\subsubsection{Concurrency gurantees}
/* I was thinking that we should add actuators to our system. For example, a sensor to detect if the light is on and is capable of turning the light on/off (or something like, if the temperature is above 25 deg. turn AC on). What is here is how to gurantee that a query is right. I propose to treat it as timestamp-based concurrency management, and the proof of correctness is available. Also, we should talk that a property of this scheme is that a correct behavior is maintained while that we always accept read requests, but a write can be rejected.
BTW, I beleive this is VEEEERY novel, they usually talk about sensing only.
 */

\subsection{Back-end design}

\subsection{Hardware}
We are using three Arduino Uno microcontrollers as end nodes in our prototype topology. Each of these microcontrollers is connected to a XBee Series 2 module for wireless communication. A fourth XBee module is directly connected to a computer and acts as the coordinator of the network.

\subsection{Software}
The host connected to the XBee controller is running a LAMP-stack, i.e. a linux machine equipped with apache, MySQL, and PHP.
/* elaborate here */


