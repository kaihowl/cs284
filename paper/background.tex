\section{Background}\label{sec:background}

\begin{figure}[t]
\centering
\includegraphics[scale=1.2]{figures/topology.eps}
\caption{Components used in the prototype}
\label{fig:topology}
\end{figure}

%/* speak here about arduino, XBee, ZigBee protocol, and 802.15.4 */
A depiction of our prototype is supplied in Figure~\ref{fig:topology}. It consists of a front-end that receives external requests and delivers readings for outside users. The back-end is a collection of sensor nodes consisting of a micro-controller and a wireless module. A detailed description of used components follows:
\begin{itemize}
\item{\emph{Arduino}: Arduino is a micro-controller capable of sensing the environment by receiving input from sensors in its surroundings. The micro-controller on the board is programmed using the Arduino programming language (based on C/C++ library) and the Arduino development environment (based on \textit{Processing}). In Arduino the connectors are exposed in such a way that allows CPU boards to be connected to a variety of interchangeable add-on modules. In our experiments we use Arduino Uno boards.}
\item{\emph{XBEE}: XBee is a radio module based on IEEE 802.15.4 networking protocol providing wireless end-point connectivity to devices.  It is designed for high-throughput applications which require low latency and predictable communication timing. AT commands are used to control the radio settings. They can operate either in a transparent data mode or in a packet based (API) mode. In the transparent mode data coming into the DIN (Data IN) pin gets directly transmitted to the intended receiving radios without any modification. Incoming packets can either be directly addressed to one target or broadcast to multiple targets. This mode is primarily used in instances where an existing protocol cannot tolerate changes to the data format. We use XBee series 2 modules.}
\item{\emph{ZigBee}: ZigBee is a specification for a suite of high level communication protocols using small, low-power digital radios based on an IEEE 802 standard for Personal Area Network (PAN). ZigBee builds upon the physical layer and medium access control defined in IEEE standard 802.15.4 (2003 version) for low-rate wireless PANs.  It is targeted at radio-frequency (RF) applications that require a low data rate, long battery life, and secure networking. ZigBee data rate varies from 20 kbps to 900 kbps but 250 kbps is best suited for periodic or intermittent data or a single signal transmission from a sensor or input device. The current ZigBee protocol supports beacon and non-beacon enabled networks. In our experiments we operate in non-beacon mode. There are three different types of Zigbee devices: 
\begin{itemize}
\item{\emph{ZigBee Coordinator (ZC)}: it forms the root of the network tree and might bridge to other networks. There is exactly one ZC per network. }
\item{\emph{ZigBee Router (ZR)}: it can run an application and can also act as an intermediate router. }
\item{\emph{ZigBee End Device (ZED)}: It has the capability to talk to the parent node. It cannot relay data from other devices. This relationship allows the node to be asleep a significant amount of the time thereby giving long battery life.}
\end{itemize}
}
\end{itemize}
